\section{Règles du jeu}
	\label{sec:regles}
	\subsection{Déroulement d'une partie}
		Le jeu se déroule au tour par tour. C'est à dire que chaque joueur va jouer à tour de rôle.
		Chaque joueur essaie d'avoir plus de points que son adversaire au terme d'un nombre de tours décidé au début de la partie.
	
	\subsection{Tours de jeu}
		Au cours d'un tour, le joueur courant déplace ses unités sur la carte autant qu'il le souhaite et le peut. Il peut éventuellement attaquer des unités adverses.
		Le tour se termine seulement lorsqu'il termine explicitement son tour (cf interface de jeu, section \ref{sec:interfaceIG})
		A la fin du tour, le joueur courant gagne des points en fonction du type de case occupé par ses unités et de leur tribu. 
		Par défaut, chaque tuile rapporte un point.
		
	\subsection{Mouvements et attaques}
		Une unité peut se déplacer sur les cases adjacentes à elle-même. Celà lui coute un certain nombre de points de mouvement.
		Toute unité commence son tour avec 4 points de mouvement. Un mouvement basique coute 2 points de mouvement, sauf spécificité de la tribu.
		L'attaque s'effectue via un déplacement.
		En effet, si une unité se déplace sur une case où se trouve une ou plusieurs unités adverse, la procédure de combat est lancée.
		\newline
		L'honneur compte dans Smallworld : l'attaquant attaque toujours au défenseur le plus fort, 
		S'en suis un combat comprenant un certain nombre de rounds; les unités ont des pv, plus il y a de rounds. Le minimum est cependant de 3 attaques de prévues contre une unité.
		Le combat peut bien sur se terminer plus tôt si l'une des deux venait à mordre la poussière !
	
	\subsection{Fin du jeu}
		Une partie dure jusqu'à ce qu'une des conditions de fin de partie soit remplie.
		\begin{itemize}
		  \item L'épuisement du compteur de tours.
		  \item Tout les adversaires d'un joueur sont mort (ils ne disposent plus d'unité).
		\end{itemize}
		A la fin de la partie, le compte final des points est affiché. On peut ainsi savoir le vainqueur de la bataille.
		Et malheur au vaincu !
		
	\subsection{Tribu}
		Dans le jeu, chaque joueur peut controler différentes tribus; voici la liste de leur spécificités.
		\begin{description}
			\item[Elfe] \hfill \\
				\textit{Leur agilité légendaire n'est plus à démontrer. Elle leur permet parfois d'échapper à la mort. Parfois ..} \\
				\begin{itemize}
					\item Cout de déplacement vers la forêt : 1
					\item Cout de déplacement vers le désert : 4
					\item 50% de chance de survivre à une attaque mortelle
				\end{itemize}
			\item[Orc] \hfill \\
				\textit{Ces guerriers sans foi ni loi se repaissent des cadavres de leurs ennemis. Tout adversaire vaincu leur fait gagner 1 point de vie} \\
				\begin{itemize}
					\item Cout de déplacement vers la plaine : 1
					\item Points sur une tuile forêt : 0
					\item Tuer une unité rapporte un point de vie
				\end{itemize}
			\item[Nain] \hfill \\
				\textit{Ces fiers guerriers ont maintes fois démontré leur maîtrise de la pioche. Il leur est aisé de se rendre d'une montagne à une autre, si le cadre est sûr.} \\
				\begin{itemize}
					\item Cout de déplacement vers la plaine : 1
					\item Points sur une tuile plaine : 0
					\item Peut aller de montagne en montagne comme un déplacement normal; ne peut lancer de combat de cette manière.
				\end{itemize}
		\end{description}